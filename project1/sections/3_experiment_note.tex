\section{实验过程记录}

\subsection{问题1:ALU计算错误}
\textbf{问题描述:}误以为运算会自动零扩展,未对num1作位数扩展,与num2相加时,结果错误。

\textbf{解决方案:}对num1进行位数扩展,使用Verilog的零扩展语句,将8位数扩展为32位:assign num1\_ext = \{24'b0, num1\};
\subsection{问题2:ALU实现各指令码所对应的功能}
\textbf{问题描述:}实现过程中,需要根据操作码(op)执行不同的运算逻辑,确保 ALU 能正确完成加法、减法、位运算和比较等功能。

\textbf{解决方案:}使用 Verilog 中的 case 语句对操作码进行译码,根据不同的值执行相应的算术或逻辑运算,例如加法、减法、按位与、
按位或、取反和无符号小于比较。通过组合逻辑块 always @(*) 实现功能控制。
\subsection{问题3:验证 ALU 各功能正确性}
\textbf{问题描述:}需要验证 ALU 模块是否按照设计正确执行加法、减法、与、或、取反、小于判断等功能。

\textbf{解决方案:}编写 Testbench 仿真文件,采用 initial 块在每个时钟上升沿提供输入数据,包括 8 位 num1、32 位 num2
 和 3 位操作码 op,通过 \$display 打印输出结果,在控制台查看各功能的执行情况并与预期对比验证正确性。
