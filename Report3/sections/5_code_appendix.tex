\appendix
\newpage
\section{Datapath代码}
\begin{lstlisting}[language=Verilog]
`timescale 1ns / 1ps

module datapath(
    input wire clka,           // 时钟
    input wire rst,            // 复位信号
    input wire jump,           // 跳转控制
    input wire branch,         // 分支控制
    input wire alusrc,         // ALU源选择
    input wire memtoreg,       // 数据选择:内存到寄存器
    input wire regwrite,       // 寄存器写控制
    input wire regdst,         // 寄存器目的选择
    input wire [2:0] alucontrol, // ALU控制信号
    output wire [31:0] aluout, // ALU结果
    output wire [31:0] pc,     // 当前PC地址
    output wire [31:0] writedata, // 写入数据
    input wire [31:0] instr,   // 指令
    input wire [31:0] readdata // 数据存储器读出数据
);

    wire [31:0] readdata1;     // 寄存器读数据1
    wire [31:0] readdata2;     // 寄存器读数据2
    wire [31:0] addrext;       // 扩展后的地址
    wire [31:0] pcnew;         // 新的PC值
    wire [31:0] pcplus4;       // PC+4
    wire [31:0] jumpaddr;      // 跳转地址
    wire [27:0] jumpleft;      // 左移后的跳转地址
    wire [31:0] branchleft;    // 左移后的分支地址
    wire [31:0] objectaddr;    // 目标地址
    wire pcsrc;                // PC源选择
    wire zero;                 // 零标志

    // 寄存器模块
    regfile regfile(
        .clk(clka),
        .rst(rst),
        .we3(regwrite),        // 修正:使用regwrite而不是memtoreg
        .ra1(instr[25:21]),
        .ra2(instr[20:16]),    // 修正:ra2应该是instr[20:16]而不是instr[15:11]
        .wa3((regdst==1) ? instr[15:11] : instr[20:16]),
        .wd3((memtoreg==1) ? readdata : aluout),
        .rd1(readdata1),
        .rd2(readdata2)
    );
    
    // 将readdata2赋给writedata,用于存储到内存
    assign writedata = readdata2;

    // 符号扩展模块
    signext sext(
        .addr(instr[15:0]),
        .y(addrext)
    );

    // ALU模块
    alu alu1(
        .num1(readdata1),
        .num2((alusrc==1) ? addrext : readdata2),
        .op(alucontrol),
        .result(aluout),
        .zero(zero)
    );

    // 分支条件 = branch & zero
    assign pcsrc = branch & zero;

    // PC模块
    pc pc1(
        .clk(clka),
        .rst(rst),
        .pcnew(pcnew),
        .pc(pc)
    );

    // 加法器1:计算PC+4
    adder ad1(
        .a(pc),
        .b(32'd4),
        .y(pcplus4)
    );

    // 加法器2:计算分支目标地址
    adder ad2(
        .a(pcplus4),
        .b(branchleft),
        .y(objectaddr)
    );

    // 26位左移2位模块
    bits26sl2 s1(
        .addr(instr[25:0]),
        .y(jumpleft)
    );

    // 计算跳转地址
    assign jumpaddr = {pcplus4[31:28], jumpleft};

    // 左移2位模块
    sl2 s2(
        .addr(addrext),
        .y(branchleft)
    );

    // 计算新的PC值
    // 如果jump=1,使用jumpaddr
    // 如果jump=0但pcsrc=1(branch & zero=1),使用objectaddr
    // 否则使用pcplus4
    assign pcnew = (jump==1'b1) ? jumpaddr : 
                  (pcsrc==1'b1) ? objectaddr : pcplus4;

endmodule
\end{lstlisting}
