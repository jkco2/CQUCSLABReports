\section{实验设计}

\subsection{Controller}\label{sub:controller}
\subsubsection{功能描述}
Controller 模块负责将取来的指令中的 OP 和 FUNC 字段翻译成各类控制信号,驱
动下游的数据通路(Datapath)完成正确的指令执行。它由Maindec和ALUdec两级译码器组成
通过这两级译码,Controller 将复杂的指令语义分解为一组简单的布尔控制信号,确保数据通路各单元在每个时钟周期内协同工作。
\subsubsection{接口定义}
\begin{table}[H]
\caption{Controller接口}\label{tab:Controller接口}
\begin{center}
	\begin{tabular}{|l|l|l|p{6cm}|}
	\hline
	\textbf{信号名} & \textbf{方向} & \textbf{位宽} & \textbf{功能描述}\\ \hline \hline
	op			& Input& 6-bit & 指令的opcode字段\\
	func&Input&6-bit&R-type指令的function字段\\
	jump&Output&1-bit&跳转控制信号,高电平表示执行J-type\\
	branch&Output&1-bit&分支判断使能,与zero相与成PCsrc控制信号\\
	alusrc&Output&1-bit&AlU第二操作数来源:0=寄存器,1=立即数\\
	alucontrol&Output&3-bit&传输给ALU的具体运算控制码\\
	memwrite&Output&1-bit&数据存储写使能,高电平写\\
	memtoreg&Output&1-bit&写回寄存器的数据来源:0=ALU,1=Mem\\
	regdst&Output&1-bit&写回寄存器来源:0=rt,1=rd\\
	regwrite&Output&1-bit&寄存器写使能,高电平写\\
	\hline
	\end{tabular}
\end{center}
\end{table}

\subsection{数据通路}\label{sub:datapath}
\subsubsection{功能描述}
Datapath 模块实现指令的取指、译码、执行、访存与回写五大阶段的数据流与运算逻辑。
\subsubsection{接口定义}
\begin{table}[H]
\caption{datapath接口}\label{tab:datapath接口}
\begin{center}
	\begin{tabular}{|l|l|l|p{6cm}|}
	\hline
	\textbf{信号名} & \textbf{方向} & \textbf{位宽} & \textbf{功能描述}\\ \hline \hline
	clka			& Input& 1-bit & 时钟信号,所有时序逻辑同步上升沿\\
	rst&Input&1-bit&异步复位,高电平清零PC和寄存器\\
	jump&Input&1-bit&来自controller的跳转控制\\
	branch&Input&1-bit&来自controller的分支控制\\
	alusrc&Input&1-bit&来自controller的ALU源选择\\
	memtoreg&Input&1-bit&来自controller的写回数据源选择\\
	regwrite&Input&1-bit&来自controller的寄存器写使能\\
	alucontrol&Input&1-bit&来自controller 的 ALU 运算控制信号\\
	instr&Input&32-bit&取指存储器输出的32位指令\\
	readdata&Input&32-bit&数据存储器读出的32位数据\\
	aluout&Output&32-bit&ALU运算结果\\
	pc&Output&32-bit&当前程序计数器值\\
	writedata&Output&32-bit&要写入数据存储器的数据总线\\
	\hline
	\end{tabular}
\end{center}
\end{table}