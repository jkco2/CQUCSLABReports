\section{实验过程记录}
\subsection{问题1:状态机设计失误}
\textbf{问题描述:}误设置WM状态在来自内存的数据未到时进入IDLE状态,导致仿真到达某一指令后锁死。

\textbf{解决方案:}WM状态在来自内存的数据未到时应保持在WM状态。
\subsection{问题2:需要flag判断read\_finish升高时完成的是读缺失的读取内存事务还是写缺失的读取内存事务}
\textbf{问题描述:}虽然指导书上说Cache在处理读取内存或写内存的时候从该CPU输入的变量cpu\_data\_req,cpu\_data\_wr,cpu\_data\_addr,
cpu\_data\_wdata,cpu\_data\_size是不变的,但是在某些CPU这些变量都是有可能改变的,为了严谨不
能靠write或read判断是写操作还是读操作。

\textbf{解决方案:}这里用一个flag信号——write\_miss\_pending进行读写操作的判断进而调控写入Cache的内容。
\subsection{问题3:写缺失后内存的写回}
\textbf{问题描述:}如果当前是写缺失情况,write\_cache\_data使用cache\_block,当有字节更新时会混入原Cache块的内容,因此写掩码生成需要内存中的当前数据而不是Cache中的数据。

\textbf{解决方案:}变更更新字节的对象,将cache\_block[index]改为cache\_data\_rdata,