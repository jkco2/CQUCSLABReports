\section{实验结果及分析}
\begin{figure}[H]
  \centering
  \begin{subfigure}[b]{0.8\textwidth}
    \centering
    \includegraphics[width=\textwidth]{image/wt_miss.png}
    \caption{写直达的写缺失}
    \label{fig:wtwm}
  \end{subfigure}
  \hfill
  \begin{subfigure}[b]{0.8\textwidth}
    \centering
    \includegraphics[width=\textwidth]{image/wb_miss.png}
    \caption{写回-写分配的写缺失}
    \label{fig:wbwm}
  \end{subfigure}
  \caption{写直达和写回-写分配的写缺失对比}
  \label{fig:miss}
\end{figure}
写回的写内存请求cache\_data\_wr在写字且原cacheline不为脏页时由于是写在cache中所以不为1。
\begin{figure}[H]
  \centering
  \begin{subfigure}[b]{0.8\textwidth}
    \centering
    \includegraphics[width=\textwidth]{image/wt.png}
    \caption{写直达}
    \label{fig:sub-a}
  \end{subfigure}
  \hfill
  \begin{subfigure}[b]{0.8\textwidth}
    \centering
    \includegraphics[width=\textwidth]{image/wb.png}
    \caption{写回-写分配}
    \label{fig:sub-b}
  \end{subfigure}
  \caption{写直达和写回-写分配的仿真对比}
  \label{fig:sim_compare}
\end{figure}
写回-写分配策略的Cache完成仿真任务比写直达策略的Cache快400微妙左右
\begin{figure}[H]
    \centering
    \includegraphics[width=0.9\textwidth]{image/pass.png}
    \caption{通过仿真测试}
    \label{fig:tcl}
\end{figure}