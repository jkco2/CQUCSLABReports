\section{实验过程记录}

\subsection{问题1:ALU计算错误}
\textbf{问题描述:}误以为运算会自动零扩展,未对num1作位数扩展,与num2相加时,结果错误。

\textbf{解决方案:}对num1进行位数扩展,使用Verilog的零扩展语句,将8位数扩展为32位:assign num1\_ext = \{24'b0, num1\};
\subsection{问题2:ALU实现各指令码所对应的功能}
\textbf{问题描述:}实现过程中,需要根据操作码(op)执行不同的运算逻辑,确保 ALU 能正确完成加法、减法、位运算和比较等功能。

\textbf{解决方案:}使用 Verilog 中的 case 语句对操作码进行译码,根据不同的值执行相应的算术或逻辑运算,例如加法、减法、按位与、
按位或、取反和无符号小于比较。通过组合逻辑块 always @(*) 实现功能控制。
\subsection{问题3:验证 ALU 各功能正确性}
\textbf{问题描述:}需要验证 ALU 模块是否按照设计正确执行加法、减法、与、或、取反、小于判断等功能。

\textbf{解决方案:}编写 Testbench 仿真文件,采用 initial 块在每个时钟上升沿提供输入数据,包括 8 位 num1、32 位 num2
 和 3 位操作码 op,通过 \$display 打印输出结果,在控制台查看各功能的执行情况并与预期对比验证正确性。

\subsection{问题4:流水线加法器仿真图难以Debug}
\textbf{问题描述:}
\begin{lstlisting}[language=Verilog]
    module adder(
        input clk,reset,
        input [3:0]refresh,
        input [3:0]stop,
        input cin,         //初始进位
        input validin,     //初始值有效
        input out_allow,    //允许输出
        input [31:0] num1,
        input [31:0] num2,
        
        output validout,
        output [31:0] ans,
        output cout,
            );
\end{lstlisting}
只使用上方代码中的接口难以看清流水线加法器内部结构,难以Debug并且难以判断功能是否正常。
\textbf{解决方案:}在上方代码中添加如下代码,添加了流水线加法器内部结构,便于Debug。
\begin{lstlisting}[language=Verilog]
    module adder(
        input clk,reset,
        input [3:0]refresh,
        input [3:0]stop,
        input cin,         //初始进位
        input validin,     //初始值有效
        input out_allow,    //允许输出
        input [31:0] num1,
        input [31:0] num2,

        output validout,
        output [31:0] ans,
        output cout,

        output debug_cout1,
        output debug_cout2,
        output debug_cout3,
        output debug_cout4,

        output [7:0] debug_sum1,
        output [15:0] debug_sum2,
        output [23:0] debug_sum3,
        output [31:0] debug_sum4,
        output pipe1_valid_out,
        output pipe2_valid_out,
        output pipe3_valid_out,
        output pipe4_valid_out
            );
\end{lstlisting}

\subsection{问题5:仿真结果不正确}
\textbf{问题描述:}由仿真结果观察,仿真结果不正确。无法做到暂停功能并且 ans 没有正确结果。

\textbf{解决方案:}仔细观察代码后发现,将部分变量的变量名写错,导致仿真结果不正确。将变量名改正后,仿真结果正确。
