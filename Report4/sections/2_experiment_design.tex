\section{实验设计}
\subsection{数据通路模块}\label{sub:datapath}
\subsubsection{功能描述}
Datapath 模块实现指令的取指、译码、执行、访存与回写五大阶段的数据流与运算逻辑。
\subsubsection{接口定义}
\begin{table}[H]
\caption{Datapath接口}\label{tab:dp}
\begin{center}
	\begin{tabular}{|l|l|l|p{6cm}|}
	\hline
	\textbf{信号名} & \textbf{方向} & \textbf{位宽} & \textbf{功能描述}\\ \hline \hline
	clk & input & 1 & 时钟信号 \\
reset & input & 1 & 同步复位信号 \\
instr & input & 32 & 当前取出的指令,来自指令存储器 \\
pc & output & 32 & 当前PC地址,供指令存储器使用 \\
memtoregD & input & 1 & ID阶段:写回阶段是否从内存读取 \\
memwriteD & input & 1 & ID阶段:是否写入内存 \\
branchD & input & 1 & ID阶段:是否为条件分支指令 \\
alusrcD & input & 1 & ID阶段:ALU第二个操作数来源选择(寄存器或立即数) \\
regdstD & input & 1 & ID阶段:写寄存器地址选择(rd或rt) \\
regwriteD & input & 1 & ID阶段:是否写寄存器 \\
jumpD & input & 1 & ID阶段:是否为跳转指令 \\
alucontrolD & input & 3 & ID阶段:ALU操作控制信号 \\
EqualD & output & 1 & ID阶段:比较rs与rt是否相等(用于分支判断) \\
StallD & output & 1 & ID阶段:是否暂停流水线(冒险检测结果) \\
pcsrcD & output & 1 & IF阶段:是否进行条件跳转(EqualD \& branchD) \\
op & output & 6 & ID阶段:指令操作码字段instr[31:26] \\
funct & output & 6 & ID阶段:R型指令功能码字段instr[5:0] \\
readdata & input & 32 & MEM阶段:从数据存储器读取的数据 \\
aluout\_M & output & 32 & MEM阶段:ALU计算结果,写数据存储器地址或写回用 \\
Writedata\_M & output & 32 & MEM阶段:写入数据存储器的数据 \\
memwriteM & output & 1 & MEM阶段:是否进行数据存储器写操作 \\
	\hline
	\end{tabular}
\end{center}
\end{table}
\subsection{冒险处理模块}\label{sub:hazard}
\subsubsection{功能描述}
Hazard模块用于检测和处理MIPS五级流水线中的数据冒险与控制冒险,通过转发(Forwarding)与暂停(Stall)机制保证数据正确性和指令正确执行。
\subsubsection{接口定义}
\begin{table}[H]
\caption{Hazard接口定义}\label{tab:signaldef}
\begin{center}
	\begin{tabular}{|l|l|l|p{6cm}|}
	\hline
	\textbf{信号名} & \textbf{方向} & \textbf{位宽} & \textbf{功能描述}\\ \hline \hline
writereg\_e & input & 5 & EX阶段要写回的寄存器号(用于检测EX阶段的RAW冒险) \\
writereg\_m & input & 5 & MEM阶段要写回的寄存器号(用于前推判断) \\
writereg\_w & input & 5 & WB阶段要写回的寄存器号(用于前推判断) \\
regwriteE & input & 1 & EX阶段写回使能信号 \\
regwriteM & input & 1 & MEM阶段写回使能信号 \\
regwriteW & input & 1 & WB阶段写回使能信号 \\
branchD & input & 1 & ID阶段当前指令是否为分支指令 \\
rsE & input & 5 & EX阶段的源寄存器rs \\
rtE & input & 5 & EX阶段的源寄存器rt \\
rsD & input & 5 & ID阶段的源寄存器rs \\
rtD & input & 5 & ID阶段的源寄存器rt \\
memtoregE & input & 1 & EX阶段指令是否从内存读取数据(lw) \\
memtoregM & input & 1 & MEM阶段指令是否从内存读取数据(lw) \\
forwardAE & output & 2 & EX阶段ALU第一个操作数的前推控制信号 \\
forwardBE & output & 2 & EX阶段ALU第二个操作数的前推控制信号 \\
forwardAD & output & 1 & ID阶段用于分支判断的rs是否需要前推 \\
forwardBD & output & 1 & ID阶段用于分支判断的rt是否需要前推 \\
StallF & output & 1 & 是否暂停IF阶段(冒险检测结果) \\
StallD & output & 1 & 是否暂停ID阶段(冒险检测结果) \\
FlushE & output & 1 & 是否清空EX阶段寄存器(防止错误执行) \\
	\hline
	\end{tabular}
\end{center}
\end{table}
