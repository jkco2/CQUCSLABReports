\section{实验过程记录}
\subsection{问题1:仿真多数据出现不定态X}
\textbf{问题描述:}refile的内部信号未初始化0,加之controller中的控制信号在某些情况规定为x,虽然逻辑上在某些情况这些控制信号是0是1都是可以的,
但是会影响到其他运算,导致不定态在运算中不断传递。

\textbf{解决方案:}将controller中的控制信号为x改成0。
\subsection{问题2:ALU无法读到数据}
\textbf{问题描述:}$or$ 指令无法读到 $addi \$2, \$0, 5$ 的数据,导致 $\$4$ 的值错误。因为regfile中 \$2的数据输入与读取在同一上升沿,故无法取到正确的数。

\textbf{解决方案:}将regfile的 $posedge clk$ 改为 $negedge clk $,使得数据提前半周期读入regfile。
\subsection{问题3:datamemory最后无法正确输出已经存入的值}
\textbf{问题描述:}通过仿真发现datamemory的输出延后实际信号memwriteM三周期。

\textbf{解决方案:}通过排查发现接口接错导致memwriteM没有正确输出,而是将memwriteE当作memwriteM输出。于是将输出接口修正。